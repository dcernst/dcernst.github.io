\documentclass[11pt]{article}

\usepackage{url}
\usepackage{array}
\usepackage{multicol}
\usepackage{tabu}
\usepackage[table]{xcolor}
\usepackage{tikz}
\usetikzlibrary{shapes.geometric}
\usepackage{fancyhdr}
\usepackage[margin=.7in]{geometry}
\usepackage[hang,flushmargin,symbol*]{footmisc}
\usepackage{amsmath}
\usepackage{amsthm}
\usepackage{amssymb}
\usepackage{mathtools}
\usepackage{enumitem}
\usepackage{graphicx}
\usepackage{color}
\definecolor{darkblue}{rgb}{0, 0, .6}
\definecolor{grey}{rgb}{.7, .7, .7}
\usepackage[breaklinks]{hyperref}

\theoremstyle{definition} 
\newtheorem{theorem}{Theorem}
\newtheorem{lemma}[theorem]{Lemma}
\newtheorem{claim}[theorem]{Claim}
\newtheorem{corollary}[theorem]{Corollary}
\newtheorem{conjecture}[theorem]{Conjecture}
\newtheorem{definition}[theorem]{Definition}
\newtheorem{example}[theorem]{Example}
\newtheorem{remark}[theorem]{Remark}
\newtheorem{important}[theorem]{Important Note}
\newtheorem{recall}[theorem]{Recall}
\newtheorem{note}[theorem]{Note}
\newtheorem{question}[theorem]{Question}
\newtheorem*{definition*}{Definition}
\newtheorem*{theorem*}{Theorem}
\newtheorem*{claim*}{Claim}

\newcommand{\ds}{\displaystyle}
\newcommand{\Spin}{\operatorname{Spin}}
\newcommand{\Rng}{\operatorname{Rng}}
\DeclareMathOperator{\range}{Rng}

\setlength{\parindent}{0pt}
\setlength{\fboxsep}{10pt}

\usepackage[linewidth=1pt]{mdframed}

%%%%%%Header/Footer%%%%%%%

\pagestyle{fancy}

\lhead{\scriptsize  MAT 320: Foundations of Mathematics - Spring 2025}
\chead{} 
\rhead{\scriptsize Final Exam (Part 2)} 
\lfoot{\scriptsize This work is licensed under the \href{https://creativecommons.org/licenses/by-sa/4.0/}{Creative Commons Attribution-Share Alike 4.0 License}.} 
\cfoot{} 
\rfoot{\scriptsize Written by \href{http://dcernst.github.io}{D.C. Ernst}} 
\renewcommand{\headrulewidth}{0.4pt} 
\renewcommand{\footrulewidth}{0.4pt}

%%%%%%%%%%%%%%%%%%%

\begin{document}

\tikzstyle{vert} = [circle, draw, fill=grey,inner sep=0pt, minimum size=6mm]
\tikzstyle{vertsmall} = [circle=3pt, draw, fill=grey]
\tikzstyle{s} = [draw, very  thick, black, stealth-stealth]
\tikzstyle{s1} = [draw, very  thick, black,-stealth]
\tikzstyle{d} = [draw, very  thick, black, dashed,-stealth]
\tikzstyle{d2} = [draw, very thick, dashed,stealth-stealth]
\tikzstyle{s2} = [draw, very thick, stealth-stealth]
\tikzstyle{snake} = [draw, very thick, snake it,-stealth]
\tikzstyle{snake2} = [draw, very thick, snake it, stealth-stealth]
\tikzstyle{g} = [draw, very thick, grey,-stealth]
\tikzstyle{g2} = [draw, very thick, grey, stealth-stealth]

\begin{center}

{\Large\bf Final Exam (Part 2)}

\bigskip

  \fbox{\parbox{7in}{
    \vspace{5pt}
    \textbf{\large Your Name:}
    \vspace{5pt}
  }}
  
  \bigskip
  
  \fbox{\parbox{7in}{
    \vspace{5pt}
    \textbf{\large Names of Any Collaborators:}
    \vspace{5pt}
  }}

\end{center}

\section*{Instructions}

Submit your solutions to the following questions by the 3\textsc{pm} on \textbf{Friday, May 9}. This part of the Final Exam is worth a total of 26 points and is worth 50\% of your overall score on the Final Exam. Your overall score on the Final Exam is worth 20\% of your overall grade.

\bigskip

I expect your solutions to be \emph{well-written, neat, and organized}.  Do not turn in rough drafts.  What you turn in should be the ``polished'' version of potentially several drafts.  Feel free to type up your final version.  The \LaTeX\ source file of this exam is also available if you are interested in typing up your solutions using \LaTeX.  I'll gladly help you do this if you'd like.

\bigskip

Reviewing material from previous courses and looking up definitions and theorems you may have forgotten is fair game. However, when it comes to completing the following problems, you should \emph{not} look to resources outside the context of this course for help.  That is, you should not be consulting the web, other texts, other faculty, or students outside of our course in an attempt to find solutions to the problems you are assigned.  This includes ChatGPT, Chegg, and Course Hero. On the other hand, you may use each other, the textbook, me, and your own intuition. Further information:
\begin{enumerate}
\item You may freely use any theorems that we have discussed in class, but you should make it clear where you are using a previous result and which result you are using.  For example, if a sentence in your proof follows from Theorem X.Y, then you should say so.
\item Unless you prove them, you cannot use any results from the course notes that we have not yet covered.
\item You are \textbf{NOT} allowed to consult external sources when working on the exam.  This includes people outside of the class, other textbooks, and online resources.
\item You are \textbf{NOT} allowed to copy someone else's work.
\item You are \textbf{NOT} allowed to let someone else copy your work.
\item You are allowed to discuss the problems with each other and critique each other's work.
\end{enumerate}

\begin{center}
\textbf{I will vigorously pursue anyone suspected of breaking these rules.}
\end{center}

You should \textbf{turn in this cover page} and all of the work that you have decided to submit. \textbf{Please write your solutions and proofs on your own paper.} To convince me that you have read and understand the instructions, sign in the box below.

\bigskip

  \fbox{\parbox{7in}{
    \vspace{5pt}
    \textbf{\large Signature:} \hfill
    \vspace{5pt}
  }}

\bigskip

Good luck and have fun!

\newpage

You may need to digest some new content in the book to complete the following problems.  We will spend some time discussing these concepts in class during the time you have available to work on the exam. However, you should not wait until I have discussed the relevant topics.  Just dig in and get started.

\bigskip

When completing each of the tasks below, you may utilize any result in the book that comes before the particular problem/theorem, regardless of whether you proved the previous result or not.

\bigskip

%\begin{mdframed}
%\textbf{Warning:} This exam is not easy! You will need to budget time to work on it. You will likely need assistance from your peers and from me. Also, it'll be very easy for me to determine whether you've cheated by using something like ChatGPT since the approach I'm asking you to take is likely not the one the Internet usually takes. Don't be tempted.
%\end{mdframed}

\setlength{\fboxsep}{2.5pt}

\begin{enumerate}

\item (2 points each) Complete Problem 8.88.
\item (1 point each) For parts (a)--(j), determine whether each of the statements in Problem 8.89 is True or False.
\item (4 points each) Prove \fbox{\textbf{two}} of the true statements in Problem 8.89(a)--(n).
\item (2 points each) Provide a counterexample for \fbox{\textbf{two}} of the false statements in Problem 8.89(a)--(n).
 
\end{enumerate}

\end{document}