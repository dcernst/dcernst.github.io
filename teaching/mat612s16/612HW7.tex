\documentclass[11pt]{scrartcl}
\usepackage[scale=1.5]{ccicons}
\usepackage[notextcomp]{kpfonts} 
\usepackage[margin=1in]{geometry}
\usepackage{amsthm,amssymb}
\usepackage{graphicx}
\usepackage{enumitem}
\usepackage{bm}
\usepackage{tabu}
\usepackage{tikz}

\usepackage{color}
\definecolor{darkblue}{rgb}{0, 0, .6}
\definecolor{grey}{rgb}{.7, .7, .7}
\usepackage[breaklinks]{hyperref}
\hypersetup{
	colorlinks=true,
	linkcolor=darkblue,
	anchorcolor=darkblue,
	citecolor=darkblue,
	pagecolor=darkblue,
	urlcolor=darkblue,
	pdftitle={},
	pdfauthor={}
}

\usepackage{fancyhdr}
\pagestyle{fancy}
\lhead{MAT 612 - Spring 2016}
\chead{}
\rhead{Due Wednesday, March 30}
\renewcommand{\headrulewidth}{.4pt}

\theoremstyle{definition}
\newtheorem{theorem}{Theorem}
\newtheorem{acknowledgement}[theorem]{Acknowledgement}
\newtheorem{algorithm}[theorem]{Algorithm}
\newtheorem{axiom}[theorem]{Axiom}
\newtheorem{case}[theorem]{Case}
\newtheorem{claim}[theorem]{Claim}
\newtheorem*{claim*}{Claim}
\newtheorem{conclusion}[theorem]{Conclusion}
\newtheorem{condition}[theorem]{Condition}
\newtheorem{conjecture}[theorem]{Conjecture}
\newtheorem{corollary}[theorem]{Corollary}
\newtheorem{criterion}[theorem]{Criterion}
\newtheorem{definition}[theorem]{Definition}
\newtheorem{example}[theorem]{Example}
\newtheorem{exercise}[theorem]{Exercise}
\newtheorem{journal}[theorem]{Journal}
\newtheorem{lemma}[theorem]{Lemma}
\newtheorem{notation}[theorem]{Notation}
\newtheorem{problem}[theorem]{Problem}
\newtheorem{proposition}[theorem]{Proposition}
\newtheorem{remark}[theorem]{Remark}
\newtheorem{solution}[theorem]{Solution}
\newtheorem{summary}[theorem]{Summary}
\newtheorem{skeleton}[theorem]{Skeleton Proof}
\newtheorem{activity}[theorem]{Activity}
\newtheorem{intuitivedef}[theorem]{Intuitive Definition}

\DeclareMathOperator{\Aut}{Aut}
\DeclareMathOperator{\Inn}{Inn}
\DeclareMathOperator{\Stab}{Stab}
\DeclareMathOperator{\Char}{Char}

\newcommand{\blankline}{\pagebreak[2]\vspace{.5\baselineskip}}

\setlength{\parindent}{0pt}

%Useful for cut and paste
%\begin{enumerate}[label=\rm{(\alph*)}]

\begin{document}

\title{Homework 7}
\subtitle{Abstract Algebra II}
\date{}

\maketitle
\thispagestyle{fancy}

Complete the following problems. Note that you should only use results that we've discussed so far this semester or last semester.

\begin{problem}
Determine the degree of the extension $\mathbb{Q}(2+\sqrt{3})/\mathbb{Q}$.
\end{problem}

\begin{problem}
Let $F$ be a field such that $\Char(F)\neq 2$.  Let $d_1,d_2\in F$ such that neither is a square in $F$. Prove that $F(\sqrt{d_1},\sqrt{d_2})$ is an extension of degree 4 over $F$ if $d_1d_2$ is not a square in $F$ and is of degree 2 otherwise.
\end{problem}

\begin{problem}
Let $K_1$ and $K_2$ be finite extensions of $F$ contained in the field $K$, and assume that both are splitting fields over $F$. Prove that $K_1\cap K_2$ is a splitting field over $F$.
\end{problem}

\begin{problem}
Prove that if $[F(\alpha):F]$ is odd, then $F(\alpha)=F(\alpha^2)$.
\end{problem}

\begin{problem}
Let $p$ be a prime and consider the polynomial $f(x)=x^p-x+a\in \mathbb{Z}/p\mathbb{Z}[x]$, where $a\neq 0$. 
\begin{enumerate}[label=\rm{(\alph*)}]
\item Prove that $f(x)$ is separable over $\mathbb{Z}/p\mathbb{Z}$.
\item Prove that every element of $\mathbb{Z}/p\mathbb{Z}$ is a root of $f(x)$.
\end{enumerate}
\end{problem}

\begin{problem}
Suppose $K$ is a field of characteristic $p$ such that $K$ contains an element that is not a $p$th power in $K$. 
\begin{enumerate}[label=\rm{(\alph*)}]
\item Prove that there exist irreducible inseparable polynomials over $K$.
\item Prove that there exist inseparable finite extensions of $K$.
\end{enumerate}
\end{problem}

\end{document}