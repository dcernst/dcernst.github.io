\documentclass[11pt]{scrartcl}
\usepackage[scale=1.5]{ccicons}
\usepackage[notextcomp]{kpfonts} 
\usepackage[margin=1in]{geometry}
\usepackage{amsthm,amssymb}
\usepackage{graphicx}
\usepackage{enumitem}
\usepackage{bm}
\usepackage{tabu}
\usepackage{tikz}

\usepackage{color}
\definecolor{darkblue}{rgb}{0, 0, .6}
\definecolor{grey}{rgb}{.7, .7, .7}
\usepackage[breaklinks]{hyperref}
\hypersetup{
	colorlinks=true,
	linkcolor=darkblue,
	anchorcolor=darkblue,
	citecolor=darkblue,
	pagecolor=darkblue,
	urlcolor=darkblue,
	pdftitle={},
	pdfauthor={}
}

\usepackage{fancyhdr}
\pagestyle{fancy}
\lhead{MAT 612 - Spring 2016}
\chead{}
\rhead{Due Wednesday, March 23}
\renewcommand{\headrulewidth}{.4pt}

\theoremstyle{definition}
\newtheorem{theorem}{Theorem}
\newtheorem{acknowledgement}[theorem]{Acknowledgement}
\newtheorem{algorithm}[theorem]{Algorithm}
\newtheorem{axiom}[theorem]{Axiom}
\newtheorem{case}[theorem]{Case}
\newtheorem{claim}[theorem]{Claim}
\newtheorem*{claim*}{Claim}
\newtheorem{conclusion}[theorem]{Conclusion}
\newtheorem{condition}[theorem]{Condition}
\newtheorem{conjecture}[theorem]{Conjecture}
\newtheorem{corollary}[theorem]{Corollary}
\newtheorem{criterion}[theorem]{Criterion}
\newtheorem{definition}[theorem]{Definition}
\newtheorem{example}[theorem]{Example}
\newtheorem{exercise}[theorem]{Exercise}
\newtheorem{journal}[theorem]{Journal}
\newtheorem{lemma}[theorem]{Lemma}
\newtheorem{notation}[theorem]{Notation}
\newtheorem{problem}[theorem]{Problem}
\newtheorem{proposition}[theorem]{Proposition}
\newtheorem{remark}[theorem]{Remark}
\newtheorem{solution}[theorem]{Solution}
\newtheorem{summary}[theorem]{Summary}
\newtheorem{skeleton}[theorem]{Skeleton Proof}
\newtheorem{activity}[theorem]{Activity}
\newtheorem{intuitivedef}[theorem]{Intuitive Definition}

\DeclareMathOperator{\Aut}{Aut}
\DeclareMathOperator{\Inn}{Inn}
\DeclareMathOperator{\Stab}{Stab}

\newcommand{\blankline}{\pagebreak[2]\vspace{.5\baselineskip}}

\setlength{\parindent}{0pt}

%Useful for cut and paste
%\begin{enumerate}[label=\rm{(\alph*)}]

\begin{document}

\title{Homework 6}
\subtitle{Abstract Algebra II}
\date{}

\maketitle
\thispagestyle{fancy}

Complete the following problems. Note that you should only use results that we've discussed so far this semester or last semester.

\begin{problem}
Let $F$ be a finite field of characteristic $p$.  Prove that $|F|=p^n$ for some positive integer $n$.
\end{problem}

\begin{problem}
Find the minimal polynomial of $1+i$ over $\mathbb{Q}$.
\end{problem}

\begin{problem}
Prove that $\mathbb{Q}(\sqrt{2}+\sqrt{3})=\mathbb{Q}(\sqrt{2},\sqrt{3})$ and find an irreducible polynomial having $\sqrt{2}+\sqrt{3}$ as a root.
\end{problem}

\begin{problem}
Suppose the degree of the extension $K/F$ is a prime $p$.  Prove that any subfield $E$ of $K$ containing $F$ is either $K$ or $F$.
\end{problem}

\begin{problem}
Suppose $F=\mathbb{Q}(\alpha_1,\ldots,\alpha_n)$ where $\alpha^2\in\mathbb{Q}$ for $i=1,\ldots,n$. Prove that $\sqrt[3]{2}\notin F$.
\end{problem}

\begin{problem}
For \textbf{three} of the following polynomials, determine the splitting field and its degree over $\mathbb{Q}$.
\begin{enumerate}[label=\rm{(\alph*)}]
\item $x^4-2$
\item $x^4+2$
\item $x^4+x^2+1$
\item $x^6-4$
\item $x^4-5x^2+6$
\end{enumerate}
\end{problem}

\begin{problem}
Let $K$ be a finite extension of $F$.  Prove that $K$ is a splitting field over $F$ iff every irreducible polynomial in $F[x]$ that has a root in $K$ splits completely in $K[x]$.
\end{problem}

\begin{problem}
Suppose $F$ is a field with the property that every polynomial $f(x)\in F[x]$ splits completely over $F$.  Prove that $F$ has no proper finite-degree extensions. \emph{Note:} An extension $K$ of $F$ is proper if $[K:F]>1$.
\end{problem}

\end{document}