\documentclass[11pt]{scrartcl}
\usepackage[scale=1.5]{ccicons}
\usepackage[notextcomp]{kpfonts} 
\usepackage[margin=1in]{geometry}
\usepackage{amsthm,amssymb}
\usepackage{graphicx}
\usepackage{enumitem}
\usepackage{bm}
\usepackage{tabu}
\usepackage{tikz}

\usepackage{color}
\definecolor{darkblue}{rgb}{0, 0, .6}
\definecolor{grey}{rgb}{.7, .7, .7}
\usepackage[breaklinks]{hyperref}
\hypersetup{
	colorlinks=true,
	linkcolor=darkblue,
	anchorcolor=darkblue,
	citecolor=darkblue,
	pagecolor=darkblue,
	urlcolor=darkblue,
	pdftitle={},
	pdfauthor={}
}

\usepackage{fancyhdr}
\pagestyle{fancy}
\lhead{MAT 612 - Spring 2016}
\chead{}
\rhead{Due Wednesday, February 10}
\renewcommand{\headrulewidth}{.4pt}

\theoremstyle{definition}
\newtheorem{theorem}{Theorem}
\newtheorem{acknowledgement}[theorem]{Acknowledgement}
\newtheorem{algorithm}[theorem]{Algorithm}
\newtheorem{axiom}[theorem]{Axiom}
\newtheorem{case}[theorem]{Case}
\newtheorem{claim}[theorem]{Claim}
\newtheorem*{claim*}{Claim}
\newtheorem{conclusion}[theorem]{Conclusion}
\newtheorem{condition}[theorem]{Condition}
\newtheorem{conjecture}[theorem]{Conjecture}
\newtheorem{corollary}[theorem]{Corollary}
\newtheorem{criterion}[theorem]{Criterion}
\newtheorem{definition}[theorem]{Definition}
\newtheorem{example}[theorem]{Example}
\newtheorem{exercise}[theorem]{Exercise}
\newtheorem{journal}[theorem]{Journal}
\newtheorem{lemma}[theorem]{Lemma}
\newtheorem{notation}[theorem]{Notation}
\newtheorem{problem}[theorem]{Problem}
\newtheorem{proposition}[theorem]{Proposition}
\newtheorem{remark}[theorem]{Remark}
\newtheorem{solution}[theorem]{Solution}
\newtheorem{summary}[theorem]{Summary}
\newtheorem{skeleton}[theorem]{Skeleton Proof}
\newtheorem{activity}[theorem]{Activity}
\newtheorem{intuitivedef}[theorem]{Intuitive Definition}

\DeclareMathOperator{\Aut}{Aut}
\DeclareMathOperator{\Inn}{Inn}
\DeclareMathOperator{\Stab}{Stab}

\newcommand{\blankline}{\pagebreak[2]\vspace{.5\baselineskip}}

\setlength{\parindent}{0pt}

%Useful for cut and paste
%\begin{enumerate}[label=\rm{(\alph*)}]

\begin{document}

\title{Homework 3}
\subtitle{Abstract Algebra II}
\date{}

\maketitle
\thispagestyle{fancy}

Complete the following problems. Note that you should only use results that we've discussed so far this semester or last semester.

\blankline

For Problems 3--7, assume that $F$ is a field.

\begin{problem}
Let $R$ be a commutative ring with 1.  Prove that a polynomial ring in more than one variable over $R$ is a PID.
\end{problem}

\begin{problem}
Consider the polynomial ring $\mathbb{Q}[x,y]$.
\begin{enumerate}[label=\rm{(\alph*)}]
\item Prove that the ideals $(x)$ and $(x,y)$ are prime in $\mathbb{Q}[x,y]$.
\item Prove that $(x,y)$ is a maximal ideal but $(x)$ is not maximal.
\item Prove that $(x,y)$ is not a principal ideal.
\end{enumerate}
\end{problem}

\begin{problem}
Prove that the rings $F[x,y]/(y^2-x)$ and $F[x,y]/(y^2-x^2)$ are not isomorphic for any field $F$.
\end{problem}

\begin{problem}
Let $f(x)\in F[x]$ such that $\deg(f(x))\geq 1$.  Prove that for each $\overline{g(x)}\in F[x]/(f(x))$ there is a unique $g_0(x)\in F[x]$ with $\deg(g_0(x))\leq n-1$ such that $\overline{g(x)}=\overline{g_0(x)}$. \emph{Note:} $\overline{g(x)}$ denotes passage to the quotient $F[x]/(f(x))$.
\end{problem}

\begin{problem}
Let $f(x)\in F[x]$.  Prove that $F[x]/(f(x))$ is a field iff $f(x)$ is irreducible.
\end{problem}

\begin{problem}
Let $F$ be a finite field.  Prove that $F[x]$ contains infinitely many primes. \emph{Hint:} Mimic one of the well-known proofs that there are infinitely many primes in the natural numbers.  
\end{problem}

\begin{problem}
Prove that the set $R$ of polynomials in $F[x]$ whose coefficient of $x$ is equal to 0 is a subring of $F[x]$ and that $R$ is is not a UFD.  \emph{Hint:} One approach is to find two distinct factorizations of $x^6$ into irreducibles.
\end{problem}

\end{document}