\documentclass[11pt]{scrartcl}
\usepackage[scale=1.5]{ccicons}
\usepackage[notextcomp]{kpfonts} 
\usepackage[margin=1in]{geometry}
\usepackage{amsthm,amssymb}
\usepackage{graphicx}
\usepackage{enumitem}
\usepackage{bm}
\usepackage{tabu}
\usepackage{tikz}

\usepackage{color}
\definecolor{darkblue}{rgb}{0, 0, .6}
\definecolor{grey}{rgb}{.7, .7, .7}
\usepackage[breaklinks]{hyperref}
\hypersetup{
	colorlinks=true,
	linkcolor=darkblue,
	anchorcolor=darkblue,
	citecolor=darkblue,
	pagecolor=darkblue,
	urlcolor=darkblue,
	pdftitle={},
	pdfauthor={}
}

\usepackage{fancyhdr}
\pagestyle{fancy}
\lhead{MAT 511 - Fall 2015}
\chead{}
\rhead{Due Wednesday, September 23}
%\lfoot{}%\scriptsize This work is licensed under the \href{http://creativecommons.org/licenses/by-sa/3.0/us/}{Creative Commons Attribution-Share Alike 3.0 License}.} 
%\cfoot{}
%\rfoot{\ccbysa}
\renewcommand{\headrulewidth}{.4pt}
%\renewcommand{\footrulewidth}{.4pt}

\theoremstyle{definition}
\newtheorem{theorem}{Theorem}
\newtheorem{acknowledgement}[theorem]{Acknowledgement}
\newtheorem{algorithm}[theorem]{Algorithm}
\newtheorem{axiom}[theorem]{Axiom}
\newtheorem{case}[theorem]{Case}
\newtheorem{claim}[theorem]{Claim}
\newtheorem*{claim*}{Claim}
\newtheorem{conclusion}[theorem]{Conclusion}
\newtheorem{condition}[theorem]{Condition}
\newtheorem{conjecture}[theorem]{Conjecture}
\newtheorem{corollary}[theorem]{Corollary}
\newtheorem{criterion}[theorem]{Criterion}
\newtheorem{definition}[theorem]{Definition}
\newtheorem{example}[theorem]{Example}
\newtheorem{exercise}[theorem]{Exercise}
\newtheorem{journal}[theorem]{Journal}
\newtheorem{lemma}[theorem]{Lemma}
\newtheorem{notation}[theorem]{Notation}
\newtheorem{problem}[theorem]{Problem}
\newtheorem{proposition}[theorem]{Proposition}
\newtheorem{remark}[theorem]{Remark}
\newtheorem{solution}[theorem]{Solution}
\newtheorem{summary}[theorem]{Summary}
\newtheorem{skeleton}[theorem]{Skeleton Proof}
\newtheorem{activity}[theorem]{Activity}
\newtheorem{intuitivedef}[theorem]{Intuitive Definition}

\DeclareMathOperator{\Aut}{Aut}

\newcommand{\blankline}{\pagebreak[2]\vspace{.5\baselineskip}}

\setlength{\parindent}{0pt}

%Useful for cut and paste
%\begin{enumerate}[label=\rm{(\alph*)}]

\begin{document}

\title{Homework 4}
\subtitle{Abstract Algebra I}
\date{}

\maketitle
\thispagestyle{fancy}

Complete the following problems. Note that you should only use results that we've discussed so far this semester.  

\begin{problem}
Let $G$ be the group determined by the table in Problem 9 on Homework 3 and consider the group $\mathbb{Z}_2\times \mathbb{Z}_2$ (where the operation is given by addition mod 2 in each component).  Prove that $G\cong \mathbb{Z}_2\times \mathbb{Z}_2$ by exhibiting a ``color-matching" of the group tables for both groups.
\end{problem}

\begin{problem}
Consider a nickel and a quarter that are side by side on a table (it doesn't really matter, but let's assume that the nickel starts on the left and the quarter is on the right and both are heads up).  Let $f$ denote the action of flipping the left coin over and let $w$ denote the action of swapping the positions of the two coins.  Let $G$ denote the group of actions on the two coins that is generated by $\{f,w\}$. That is, $G=\langle f,w\rangle$.
\begin{enumerate}[label=\rm{(\alph*)}]
\item Draw the Cayley diagram for $G$ using the generating set $\{f,w\}$. \emph{Hint:} There are 8 possible rearrangements of the coins and we can obtain all 8 using combinations of $f$ and $w$.
\item Using your Cayley diagram as justification, to what group is $G$ isomorphic?
\end{enumerate}
\end{problem}

\begin{problem}
Let $\phi:G_1\to G_2$ be a homomorphism.
\begin{enumerate}[label=\rm{(\alph*)}]
\item Prove that $\phi(e_1)=e_2$, where $e_i$ is the identity of $G_i$.
\item Prove that $\phi(x^n)=\phi(x)^n$ for all $x\in G_1$ and all $n\in\mathbb{Z}^+$.
\item Prove that $\phi(x^{-1})=\phi(x)^{-1}$ for all $x\in G_1$.
\item Prove that $\phi(x^n)=\phi(x)^n$ for all $x\in G$ and all $n\in\mathbb{Z}$.
\end{enumerate}
\end{problem}

\begin{problem}
Let $\phi:G_1\to G_2$ be an isomorphism.
\begin{enumerate}[label=\rm{(\alph*)}]
\item Prove that $|\phi(x)|=|x|$ for all $x\in G_1$.
\item Prove that $G_1$ and $G_2$ have the same number of elements of order $n$ for each $n\in\mathbb{Z}^+$.
\item Is part (b) true if $\phi$ is only assumed to be a homomorphism?  Justify your answer.\end{enumerate}
\end{problem}

\begin{problem}
Complete any 3 of the following.
\begin{enumerate}[label=\rm{(\alph*)}]
\item Prove that $\mathbb{Z}_2\times \mathbb{Z}_2$ is not isomorphic to $\mathbb{Z}_4$.
\item Prove that $D_8$ and $Q_8$ are not isomorphic.
\item Prove that the multiplicative groups $\mathbb{R}\setminus\{0\}$ and $\mathbb{C}\setminus\{0\}$ are not isomorphic.
\item Prove that the additive groups $\mathbb{R}$ and $\mathbb{Q}$ are not isomorphic.
\item Prove that the additive groups $\mathbb{Z}$ and $\mathbb{Q}$ are not isomorphic.
\end{enumerate}
\end{problem}

\begin{problem}
For each of the following pairs of groups, determine whether the given bijective function is an isomorphism from the first group to the second group. Justify your answers. (You may assume that each function is a bijection.)
\begin{enumerate}[label=\rm{(\alph*)}]
\item $(\mathbb{Z},+)$ and $(\mathbb{Z},+)$, $\phi(n)=n+1$.
\item $(\mathbb{Z},+)$ and $(\mathbb{Z},+)$, $\phi(n)=-n$.
\item $(\mathbb{Q},+)$ and $(\mathbb{Q},+)$, $\phi(x)=x/2$.
\end{enumerate}
\end{problem}

%\begin{problem}
%Prove that any two groups of order 2 are isomorphic. \emph{Hint:} There are lots of ways to attack this problem, but showing that the group tables match up is one way.
%\end{problem}
%
%\begin{problem}
%Prove that any two groups of order 3 are isomorphic.
%\end{problem}

\begin{problem}
Prove that the groups $(\mathbb{R},+)$ and $(\mathbb{R}^+,\cdot)$ are isomorphic.
\end{problem}

%\begin{problem}
%Prove that the groups $(\mathbb{Z},+)$ and $(2\mathbb{Z},+)$ are isomorphic.
%\end{problem}

\begin{problem}
Let $\phi:G_1\to G_2$ be a homomorphism.
\begin{enumerate}[label=\rm{(\alph*)}]
\item Prove if $\phi$ is an isomorphism, then $G_1$ is abelian iff $G_2$ is abelian.
\item What conditions on $\phi$ are sufficient to ensure that if $G_1$ is abelian, then $G_2$ is abelian, as well.  Justify your answer.
\end{enumerate}
\end{problem}

%\begin{problem}
%Let $A$ and $B$ be groups.  Prove that $A\times B\cong B\times A$.
%\end{problem}

\begin{problem}
Let $\phi:G_1\to G_2$ be a homomorphism.  Prove that the image $\phi(G_1)$ is a subgroup of $G_2$. Quickly deduce that if $\phi$ is an injection, then $G_1\cong \phi(G_1)$. 
\end{problem}

\begin{problem}
Let $\phi:G_1\to G_2$ be a homomorphism. Define the \emph{kernel} of $\phi$ to be the set
\[
\ker(\phi)=\{g\in G_1\mid \phi(g)=e_2\},
\]
where $e_2$ is the identity in $G_2$.
\begin{enumerate}[label=\rm{(\alph*)}]
\item Why is $\ker(\phi)$ always nonempty?
\item Prove that $\ker(\phi)$ is a subgroup of $G_1$.
\item Prove that $\phi$ is an injection iff $\ker(\phi)=\{e_1\}$, where $e_1$ is the identity in $G_1$.
\end{enumerate}
\end{problem}

\begin{problem}
Let $G$ be a group and let $\Aut(G)$  be the set of all isomorphisms from $G$ onto $G$.  Prove that $\Aut(G)$ is a group under function composition.  Note that $\Aut(G)$ is called the \emph{automorphism group of $G$} and its elements are referred to as \emph{automorphisms}.  There are some details to be shown for this problem, so don't cut too many corners.
\end{problem}

\end{document}