\documentclass[11pt]{article}%{scrartcl}
\usepackage[scale=1.5]{ccicons}
\usepackage[notextcomp]{kpfonts} 
\usepackage[margin=1in]{geometry}
\usepackage{amsthm,amssymb}
\usepackage{graphicx}
\usepackage{mathtools}
\usepackage{enumitem}
\usepackage{bm}
%\usepackage{ulem}
\usepackage{tikz}

\usepackage{color}
\definecolor{darkblue}{rgb}{0, 0, .6}
\definecolor{grey}{rgb}{.7, .7, .7}
\usepackage[breaklinks]{hyperref}
\hypersetup{
	colorlinks=true,
	linkcolor=darkblue,
	anchorcolor=darkblue,
	citecolor=darkblue,
	pagecolor=darkblue,
	urlcolor=darkblue,
	pdftitle={},
	pdfauthor={}
}

\usepackage{fancyhdr}
\pagestyle{fancy}
\lhead{MAT 526 - Spring 2021}
\chead{}
\rhead{Due Monday, February 22}
%\lfoot{}%\scriptsize This work is licensed under the \href{http://creativecommons.org/licenses/by-sa/3.0/us/}{Creative Commons Attribution-Share Alike 3.0 License}.} 
%\cfoot{}
%\rfoot{\ccbysa}
\renewcommand{\headrulewidth}{.4pt}
%\renewcommand{\footrulewidth}{.4pt}

\theoremstyle{definition}
\newtheorem{theorem}{Theorem}
\newtheorem{acknowledgement}[theorem]{Acknowledgement}
\newtheorem{algorithm}[theorem]{Algorithm}
\newtheorem{axiom}[theorem]{Axiom}
\newtheorem{case}[theorem]{Case}
\newtheorem{claim}[theorem]{Claim}
\newtheorem*{claim*}{Claim}
\newtheorem{conclusion}[theorem]{Conclusion}
\newtheorem{condition}[theorem]{Condition}
\newtheorem{conjecture}[theorem]{Conjecture}
\newtheorem{corollary}[theorem]{Corollary}
\newtheorem{criterion}[theorem]{Criterion}
\newtheorem{definition}[theorem]{Definition}
\newtheorem{example}[theorem]{Example}
\newtheorem{exercise}[theorem]{Exercise}
\newtheorem{journal}[theorem]{Journal}
\newtheorem{lemma}[theorem]{Lemma}
\newtheorem{notation}[theorem]{Notation}
\newtheorem{problem}[theorem]{Problem}
\newtheorem{proposition}[theorem]{Proposition}
\newtheorem{remark}[theorem]{Remark}
\newtheorem{solution}[theorem]{Solution}
\newtheorem{summary}[theorem]{Summary}
\newtheorem{skeleton}[theorem]{Skeleton Proof}
\newtheorem{activity}[theorem]{Activity}
\newtheorem{intuitivedef}[theorem]{Intuitive Definition}

\DeclareMathOperator{\des}{des}
\DeclareMathOperator{\Inv}{Inv}
\DeclareMathOperator{\inv}{inv}
\DeclareMathOperator{\Dyck}{Dyck}

\newcommand{\blankline}{\pagebreak[2]\vspace{.5\baselineskip}}

\newcommand{\euler}[2]{
  \displaystyle \left\langle\begin{matrix}#1  \\#2  \\ \end{matrix}\right\rangle}
\newcommand{\stirling}[2]{
  \displaystyle \left\{\begin{matrix}#1  \\#2  \\ \end{matrix}\right\}}

\setlength{\parindent}{0pt}

%Useful for cut and paste
%\begin{enumerate}[label=\rm{(\alph*)}]

\begin{document}

\begin{center}
{\Large\bf Homework 8}

\smallskip

Combinatorics
\end{center}

\thispagestyle{fancy}

You are allowed and encouraged to work together on homework. Yet, each student is expected to turn in their own work. In general, late homework will not be accepted. However, you are allowed to turn in \textbf{up to two late homework assignments with no questions asked}. When doing your homework, I encourage you to consult the \href{http://danaernst.com/teaching/ElementsOfStyle.pdf}{Elements of Style for Proofs}. Unless otherwise indicated, submit each of the following assignments via BbLearn by the due date. You will need to capture your handwritten work digitally and then upload a PDF to BbLearn. There are many free smartphone apps for doing this. I use TurboScan on my iPhone.

\blankline

Reviewing material from previous courses and looking up definitions and theorems you may have forgotten is fair game. However, when it comes to completing assignments for this course, you should \emph{not} look to resources outside the context of this course for help.  That is, you should not be consulting the web, other texts, other faculty, or students outside of our course in an attempt to find solutions to the problems you are assigned.  This includes Chegg and Course Hero. On the other hand, you may use each other, the textbook, me, and your own intuition. \textbf{If you feel you need additional resources, please come talk to me and we will come up with an appropriate plan of action.} Please read NAU's \href{https://www5.nau.edu/policies/Client/Details/828?whoIsLooking=Students&pertainsTo=All&sortDirection=Ascending&page=1}{Academic Integrity Policy}.

\blankline

Complete the following problems. 
\begin{enumerate}
\item Recall the definition of the Bell numbers given in Problem 2 on Homework 5. Define the exponential generating function for the Bell numbers via
\[
B(t)\coloneqq \sum_{k\geq 0}B_k\frac{t^k}{k!}.
\]
\begin{enumerate}
\item Prove that $B'(t)=e^tB(t)$.
\item Notice that the equation in part (a) is a differential equation. Taking for granted that we can formally integrate as expected, solve this differential equation to obtain a closed form for the exponential generating function for the Bell numbers.
\end{enumerate}
\item Complete Problem 1.15 from textbook.
\item Recall the definition of the Mahonian numbers given in Problem 1 on Homework 7.  Prove that the number of permutations in $S_n$ with $k$ inversions equals the number of permutations in $S_n$ with $\binom{n}{2}-k$ inversions.  That is, prove that for $n\geq 1$ and any $0\leq k\leq \binom{n}{2}$, we have
\[
I_{n,k}=I_{n,\binom{n}{2}-k}.
\]
\item A \textbf{Dyck path} of length $2n$ is a lattice path from $(0,0)$ to $(n,n)$ consisting of $n$ horizontal steps ``East" from $(i, j)$ to $(i + 1, j)$ and $n$ vertical steps ``North" from $(i, j)$ to $(i, j + 1)$, such that all points on the path satisfy $i\leq j$, i.e., the path, when drawn in the Cartesian plane, lies on or above the line $y = x$. Let $\Dyck(n)$ denote the collection of all Dyck paths of length $2n$.
 \begin{enumerate}
\item Draw all Dyck paths in $\Dyck(1)$, $\Dyck(2)$, $\Dyck(3)$, and $\Dyck(4)$.
\item Prove that $\displaystyle |\Dyck(n)|=\binom{2n}{n}-\binom{2n}{n-1}$.
\item Using the formula for computing binomial coefficients, prove that
\[
\binom{2n}{n}-\binom{2n}{n-1}=\frac{1}{n+1}\binom{2n}{n}.
\]
\end{enumerate}
It follows from parts (b) and (c) that $|\Dyck(n)|=\displaystyle \frac{1}{n+1}\binom{2n}{n}$.
\item Complete Problem 2.1 from textbook.
\item Complete Problem 2.3 from textbook.
\item Complete Problem 2.6 from textbook.
\end{enumerate}
\end{document}