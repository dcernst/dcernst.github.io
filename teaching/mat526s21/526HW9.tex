\documentclass[11pt]{article}%{scrartcl}
\usepackage[scale=1.5]{ccicons}
\usepackage[notextcomp]{kpfonts} 
\usepackage[margin=1in]{geometry}
\usepackage{amsthm,amssymb}
\usepackage{graphicx}
\usepackage{mathtools}
\usepackage{enumitem}
\usepackage{bm}
%\usepackage{ulem}
\usepackage{tikz}

\usepackage{color}
\definecolor{darkblue}{rgb}{0, 0, .6}
\definecolor{grey}{rgb}{.7, .7, .7}
\usepackage[breaklinks]{hyperref}
\hypersetup{
	colorlinks=true,
	linkcolor=darkblue,
	anchorcolor=darkblue,
	citecolor=darkblue,
	pagecolor=darkblue,
	urlcolor=darkblue,
	pdftitle={},
	pdfauthor={}
}

\usepackage{fancyhdr}
\pagestyle{fancy}
\lhead{MAT 526 - Spring 2021}
\chead{}
\rhead{Due Monday, March 15}
%\lfoot{}%\scriptsize This work is licensed under the \href{http://creativecommons.org/licenses/by-sa/3.0/us/}{Creative Commons Attribution-Share Alike 3.0 License}.} 
%\cfoot{}
%\rfoot{\ccbysa}
\renewcommand{\headrulewidth}{.4pt}
%\renewcommand{\footrulewidth}{.4pt}

\theoremstyle{definition}
\newtheorem{theorem}{Theorem}
\newtheorem{acknowledgement}[theorem]{Acknowledgement}
\newtheorem{algorithm}[theorem]{Algorithm}
\newtheorem{axiom}[theorem]{Axiom}
\newtheorem{case}[theorem]{Case}
\newtheorem{claim}[theorem]{Claim}
\newtheorem*{claim*}{Claim}
\newtheorem{conclusion}[theorem]{Conclusion}
\newtheorem{condition}[theorem]{Condition}
\newtheorem{conjecture}[theorem]{Conjecture}
\newtheorem{corollary}[theorem]{Corollary}
\newtheorem{criterion}[theorem]{Criterion}
\newtheorem{definition}[theorem]{Definition}
\newtheorem{example}[theorem]{Example}
\newtheorem{exercise}[theorem]{Exercise}
\newtheorem{journal}[theorem]{Journal}
\newtheorem{lemma}[theorem]{Lemma}
\newtheorem{notation}[theorem]{Notation}
\newtheorem{problem}[theorem]{Problem}
\newtheorem{proposition}[theorem]{Proposition}
\newtheorem{remark}[theorem]{Remark}
\newtheorem{solution}[theorem]{Solution}
\newtheorem{summary}[theorem]{Summary}
\newtheorem{skeleton}[theorem]{Skeleton Proof}
\newtheorem{activity}[theorem]{Activity}
\newtheorem{intuitivedef}[theorem]{Intuitive Definition}

\DeclareMathOperator{\des}{des}
\DeclareMathOperator{\Des}{Des}
\DeclareMathOperator{\Inv}{Inv}
\DeclareMathOperator{\inv}{inv}
\DeclareMathOperator{\area}{area}
\DeclareMathOperator{\Dyck}{Dyck}

\newcommand{\blankline}{\pagebreak[2]\vspace{.5\baselineskip}}

\newcommand{\euler}[2]{
  \displaystyle \left\langle\begin{matrix}#1  \\#2  \\ \end{matrix}\right\rangle}
\newcommand{\stirling}[2]{
  \displaystyle \left\{\begin{matrix}#1  \\#2  \\ \end{matrix}\right\}}

\setlength{\parindent}{0pt}

%Useful for cut and paste
%\begin{enumerate}[label=\rm{(\alph*)}]

\begin{document}

\begin{center}
{\Large\bf Homework 9}

\smallskip

Combinatorics
\end{center}

\thispagestyle{fancy}

You are allowed and encouraged to work together on homework. Yet, each student is expected to turn in his or her own work. In general, late homework will not be accepted. However, you are allowed to turn in \textbf{up to two late homework assignments with no questions asked}. When doing your homework, I encourage you to consult the \href{http://danaernst.com/teaching/ElementsOfStyle.pdf}{Elements of Style for Proofs}. Unless otherwise indicated, submit each of the following assignments via BbLearn by the due date. You will need to capture your handwritten work digitally and then upload a PDF to BbLearn. There are many free smartphone apps for doing this. I use TurboScan on my iPhone.

\blankline

Reviewing material from previous courses and looking up definitions and theorems you may have forgotten is fair game. However, when it comes to completing assignments for this course, you should \emph{not} look to resources outside the context of this course for help.  That is, you should not be consulting the web, other texts, other faculty, or students outside of our course in an attempt to find solutions to the problems you are assigned.  This includes Chegg and Course Hero. On the other hand, you may use each other, the textbook, me, and your own intuition. \textbf{If you feel you need additional resources, please come talk to me and we will come up with an appropriate plan of action.} Please read NAU's \href{https://www5.nau.edu/policies/Client/Details/828?whoIsLooking=Students&pertainsTo=All&sortDirection=Ascending&page=1}{Academic Integrity Policy}.

\blankline

Complete the following problems. 
\begin{enumerate}
\item Prove that a permutation $w\in S_n$ is uniquely determined by its inversion set $\Inv(w)$.
\item Given a lattice path $p$ consisting only of north and east steps, define $\area(p)$ to be the total area under the path $p$ (and above the $x$-axis). Recall the definition of $L(k,n-k)$ that appeared on Part 2 of Exam 1. Prove that
\[
\sum_{p\in L(k,n-k)}q^{\area(p)}=\sum_{\substack{w\in S_n \\ \Des(w)\subseteq \{k\}}} q^{\inv(w)}.
\]
%\item Prove that $N_{n,k}=N_{n,n-k-1}$ for $n\geq 1$ and $0\leq k\leq n-1$. \emph{Note:} We mentioned this result a few times, but never wrote down a proof.  You might find that it's a little trickier to get just right than you expect. We used this fact when deriving the closed form for $N_{n,k}$, so you cannot use that result here.
\item In class, we described a bijection $\psi:S_n(231)\to \Dyck(n)$ (also see Section~2.4.3 in book) that mapped maximal decreasing runs to peaks. However, we omitted two important details.
\begin{enumerate}
\item Explain why the path we obtain from a $231$-avoiding permutation prior to reflecting over the line $y=x$ never goes above the line $y=x$.
\item Explain why the permutation we obtain when reversing the map $\psi$ will yield a 231-avoiding permutation.
\end{enumerate}
\item Complete Problem 2.4 in textbook.
\item Complete Problem 2.5 in textbook.
\item Complete any \emph{two} of Problem 2.7, 2.8, 2.9 in textbook.
\item Find an explicit bijection between triangulations of a convex $(n+2)$-gon and $PB(n)$.
\end{enumerate}
\end{document}