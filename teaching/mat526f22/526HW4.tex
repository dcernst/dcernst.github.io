\documentclass[11pt]{article}%{scrartcl}
\usepackage[scale=1.5]{ccicons}
\usepackage[notextcomp]{kpfonts} 
\usepackage[margin=1in]{geometry}
\usepackage{amsthm,amssymb}
\usepackage{graphicx}
\usepackage{enumitem}
\usepackage{bm}

\usepackage{color}
\definecolor{darkblue}{rgb}{0, 0, .6}
\definecolor{grey}{rgb}{.7, .7, .7}
\usepackage[breaklinks]{hyperref}
\hypersetup{
	colorlinks=true,
	linkcolor=darkblue,
	anchorcolor=darkblue,
	citecolor=darkblue,
	pagecolor=darkblue,
	urlcolor=darkblue,
	pdftitle={},
	pdfauthor={}
}

\usepackage{fancyhdr}
\pagestyle{fancy}
\lhead{MAT 526 - Fall 2022}
\chead{}
\rhead{Due Wednesday, September 14}
\renewcommand{\headrulewidth}{.4pt}

\theoremstyle{definition}
\newtheorem{theorem}{Theorem}
\newtheorem{acknowledgement}[theorem]{Acknowledgement}
\newtheorem{algorithm}[theorem]{Algorithm}
\newtheorem{axiom}[theorem]{Axiom}
\newtheorem{case}[theorem]{Case}
\newtheorem{claim}[theorem]{Claim}
\newtheorem*{claim*}{Claim}
\newtheorem{conclusion}[theorem]{Conclusion}
\newtheorem{condition}[theorem]{Condition}
\newtheorem{conjecture}[theorem]{Conjecture}
\newtheorem{corollary}[theorem]{Corollary}
\newtheorem{criterion}[theorem]{Criterion}
\newtheorem{definition}[theorem]{Definition}
\newtheorem{example}[theorem]{Example}
\newtheorem{exercise}[theorem]{Exercise}
\newtheorem{journal}[theorem]{Journal}
\newtheorem{lemma}[theorem]{Lemma}
\newtheorem{notation}[theorem]{Notation}
\newtheorem{problem}[theorem]{Problem}
\newtheorem{proposition}[theorem]{Proposition}
\newtheorem{remark}[theorem]{Remark}
\newtheorem{solution}[theorem]{Solution}
\newtheorem{summary}[theorem]{Summary}
\newtheorem{skeleton}[theorem]{Skeleton Proof}
\newtheorem{activity}[theorem]{Activity}
\newtheorem{intuitivedef}[theorem]{Intuitive Definition}

\newcommand{\blankline}{\pagebreak[2]\vspace{.5\baselineskip}}

\setlength{\parindent}{0pt}

%Useful for cut and paste
%\begin{enumerate}[label=\rm{(\alph*)}]

\begin{document}

\begin{center}
{\Large\bf Homework 4}

\smallskip

Combinatorics
\end{center}

\thispagestyle{fancy}

You are allowed and encouraged to work together on homework. Yet, each student is expected to turn in his or her own work. In general, late homework will not be accepted. However, you are allowed to turn in \textbf{up to two late homework assignments with no questions asked}. 

\blankline

Reviewing material from previous courses and looking up definitions and theorems you may have forgotten is fair game. However, when it comes to completing assignments for this course, you should \emph{not} look to resources outside the context of this course for help.  That is, you should not be consulting the web, other texts, other faculty, or students outside of our course in an attempt to find solutions to the problems you are assigned.  This includes Chegg and Course Hero. On the other hand, you may use each other, Discord, me, and your own intuition. \textbf{If you feel you need additional resources, please come talk to me and we will come up with an appropriate plan of action.} Please read NAU's \href{https://www5.nau.edu/policies/Client/Details/828?whoIsLooking=Students&pertainsTo=All&sortDirection=Ascending&page=1}{Academic Integrity Policy}.

\blankline

Complete the following problems. 

\begin{enumerate}
\item Consider a $1\times n$ array of the numbers 1 through $n$. Suppose we have tiles of size $1\times 1$ and $1\times 2$ such that the tiles cover exactly one and two numbers of our array, respectively.  Let $F_n$ denote the number of tilings.
\begin{enumerate}
\item Prove that $F_n=f_n$ (where $f_n$ is the $n$th Fibonacci number).
\item Prove that for $m\geq 3$ and $n\geq 2$, we have $F_{m+n-1}=F_{m-2}F_{n-1}+F_{m-1}F_{n}$. \textit{Hint:} By definition, the lefthand side counts the number of tilings of an array with $m+n-1$ entries.  So, it suffices to show that the righthand side counts the same thing.  Number the entries 1 through $m+n-1$, from left to right. Let $\mathcal{S}_m$ be the collection of tilings where there is a $1\times 2$ tile covering the entries labeled by $m-1$ and $m$, and let $\mathcal{T}_m$ be the collection of tilings where this is not the case.  
\item What does Part~(b) tell us about the Fibonacci sequence?
\end{enumerate}

\item Show that the Fibonacci numbers satisfy the following identity:
\[
f_n = \sum_{k\geq 0} \binom{n-k}{k}.
\]
\emph{Hint:} There are at least two natural approaches.  One method would be using Pascal's Recurrence.  A second, more elegant method perhaps, would be to utilize a combinatorial argument with one of the compositions in Problem~5 on Homework~3.
\item  \emph{This problem has been removed.}%Fix $0<k\leq n$. How many injections are there of the form $w: \{1,2,\ldots,k\}\to \{1,2,\ldots,n\}$?
\item Prove that $P(n,n) = P(n,k)P(n-k,n-k)$ by using the meaning of $k$-permutations and the bijection principle. \emph{Hint:} The product principle gives us $|S_{n,k}\times S_{n-k}|=P(n,k)P(n-k,n-k)$. It suffices to describe a bijection
$f: S_n \to S_{n,k} \times S_{n-k}$.
\item Prove that $P(n,k)=P(n-1,k)+kP(n-1,k-1)$ by using the meaning of $k$-permutations and the bijection principle.
\item What are the alternating row sums in Pascal's Triangle?  That is, for $n\geq 0$, find a formula for $\sum_{k=0}^{n}(-1)^k\binom{n}{k}$. Instead of using the Binomial Theorem, find a proof that either uses the meaning of $\binom{n}{k}$ or rearrange the sum and use the symmetry theorem (i.e., $\binom{n}{k}=\binom{n}{n-k}$). You might want to consider the cases for $n$ odd versus even separately.
\item Prove that for any $k$ and $m$ less than or equal to $n$, we have $\binom{n}{k}=\sum_{j=0}^k\binom{n-m}{j}\binom{m}{k-j}$. \textit{Hint:} Split $[n]$ into two piles, say $A_m=\{1,\ldots,m\}$ and $B_m=\{m+1,\ldots,n\}$. For each $j\in\{0,\ldots, k\}$, count number of ways to get a $k$-subset by selecting the appropriate number from $A_m$ and the appropriate number from $B_m$.
\item Use the Binomial Theorem to find a formula for each of the following:
\begin{enumerate}
\item $\displaystyle \sum_{k=0}^n2^k\binom{n}{k}$
\item $\displaystyle \sum_{k=0}^n(-2)^k\binom{n}{k}$
\end{enumerate}
\end{enumerate}

\end{document}