\documentclass[11pt]{article}

\usepackage{url} 
\usepackage{tikz}
\usepackage{fancyhdr}
\usepackage[margin=.75in]{geometry}
\usepackage{subfig}
\usepackage[hang,flushmargin,symbol*]{footmisc}
\usepackage{amsmath}
\usepackage{amsthm}
\usepackage{amssymb}
\usepackage{mathtools}
\usepackage{enumitem}
\usepackage{graphicx}
\usepackage{color}
\definecolor{darkblue}{rgb}{0, 0, .6}
\definecolor{grey}{rgb}{.7, .7, .7}
\usepackage[breaklinks]{hyperref}
\hypersetup{
	colorlinks=true,
	linkcolor=darkblue,
	anchorcolor=darkblue,
	citecolor=darkblue,
	pagecolor=darkblue,
	urlcolor=darkblue,
	pdftitle={},
	pdfauthor={}
}

\theoremstyle{definition} 
\newtheorem{theorem}{Theorem}

\setlength{\parindent}{0pt}

%%%%%%Header/Footer%%%%%%%

\pagestyle{fancy}

\lhead{\scriptsize Exam 2 (take-home portion)} 
\chead{} 
\rhead{\scriptsize \thepage} 
\lfoot{\scriptsize This work is licensed under the \href{http://creativecommons.org/licenses/by-sa/3.0/us/}{Creative Commons Attribution-Share Alike 3.0 License}.} 
\cfoot{} 
\rfoot{\scriptsize Written by \href{http://oz.plymouth.edu/~dcernst}{D.C. Ernst}} 
\renewcommand{\headrulewidth}{0.4pt} 
\renewcommand{\footrulewidth}{0.4pt} 

%%%%%%%%%%%%%%%%%%%

\begin{document}

\begin{center}

{\Large\bf MA 4140: Algebraic Structures (Spring 2010)}\\
\smallskip
{\Large\bf Exam 2 (take-home portion)}

\bigskip

  \fbox{\parbox{7in}{
    \vspace{12pt}
    \textbf{\large NAME:}
    \vspace{12pt}
  }}

\end{center}

\setlength{\fboxsep}{10pt}

\section*{Instructions}

Prove any \emph{three} of the following theorems.

\bigskip

This portion of Exam 2 is worth 40 points.  Each of the three proofs that you complete is worth 10 points.  Your written presentation of the proofs (which includes spelling, grammar, punctuation, clarity, and legibility) is worth the remaining 10 points.

\bigskip

I expect your proofs to be \emph{well-written, neat, and organized}.  You should write in \emph{complete sentences}.  Do not turn in rough drafts.  What you turn in should be the ``polished'' version of potentially several drafts.  Feel free to type up your final version.  

\bigskip

The \LaTeX\ source file of this exam is also available if you are interested in typing up your solutions using \LaTeX.  I'll be happy to help you do this.

\bigskip

The simple rules for this portion of the exam are:

\begin{enumerate}
\item You may freely use any theorems that we have discussed in class, but you should make it clear where you are using a previous result and which result you are using.  For example, if a sentence in your proof follows from Proposition 2.9, then you should say so.
\item You cannot use any results from the book or otherwise that we have not covered, unless you prove them.
\item You are NOT allowed to copy someone else's work.
\item You are NOT allowed to let someone else copy your work.
\item You are allowed to discuss the problems with each other and critique each other's work.
\end{enumerate}

This portion of Exam 2 is due by 5\textsc{pm} on Friday, April 23.  You should turn in this cover page and the three proofs that you have decided to submit.

\bigskip

To convince me that you have read and understand the instructions, sign in the box below.

\bigskip

  \fbox{\parbox{7in}{
    \vspace{12pt}
    \textbf{\large Signature:} \hfill
    \vspace{12pt}
  }}

\bigskip

Good luck and have fun!

\newpage

\begin{theorem}
Let $G$ be a group and let $H\leq G$ such that all of the left cosets of $H$ are equal to the right cosets of $H$.  Then for all $a,b\in G$, if $x\in aH$ and $y\in bH$, then $xy\in abH$.
\end{theorem}

\newpage

\begin{theorem}
Let $G$ be a group and let $H$ and $K$ be subgroups of $G$ such that for all $g\in G$, $gH=Hg$ and $gK=Kg$.  Then for all $g\in G$, $g(H\cap K)=(H\cap K)g$.\footnote{Recall that $H\cap K$ is a subgroup by Exercise 2.43.}
\end{theorem}

\newpage

\begin{theorem}
Let $(G,\cdot)$ and $(H,\circ)$ be two groups.  Suppose $f:G\to H$ satisfies $f(a\cdot b)=f(a)\circ f(b)$ for all $a,b\in G$.  Define $K=\{a\in G:f(a)=e'\}$, where $e'$ is the identity in $H$.  Then $K\leq G$.\footnote{It is important to notice that we are \emph{not} assuming that $f$ is an isomorphism, but only that $f$ respects the operations of both groups.}
\end{theorem}

\newpage

\begin{theorem}
Let $(G,\cdot)$ and $(H,\circ)$ be isomorphic groups.  If $G$ has a subgroup of order $n$, then $H$ must also have a subgroup of order $n$.
\newpage
\end{theorem}

\newpage

\begin{theorem}
Let $G$ be a group and let $g\in G$.  Define $\psi: G\to G$ via
\[
\psi(x)=gxg^{-1}
\]
for all $x\in G$.  Then $\psi$ is an isomorphism of $G$ to itself.
\end{theorem}

\end{document}
