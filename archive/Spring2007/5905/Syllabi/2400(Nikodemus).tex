
%###############################################################################

  \documentclass[10pt]{article}

  \addtolength{\topmargin}{-0.60in}
  \addtolength{\textheight}{2\baselineskip}
  \setlength{\oddsidemargin}{0pt}
  \setlength{\evensidemargin}{0pt}
  \setlength{\textwidth}{6.5in}
  \setlength{\textheight}{9.5in}
  \setlength{\parindent}{0pt}
  \setlength{\leftmargini}{10pt}
% \renewcommand{\baselinestretch}{1.1}
  \pagestyle{myheadings}
  \markright{\sc MATH2400 --- Summer 2006 --- M. Nickodemus}

%###############################################################################

\begin{document}

%###############################################################################

\thispagestyle{plain}

\addtolength{\topmargin}{-0.40in}
\centerline{
  \framebox{\parbox{4.0in}{
    \vspace{6pt}
      \begin{centering}
             \Large       \bf C~A~L~C~U~L~U~S~~~~3            \\
             \normalsize  \bf (~M~A~T~H~~~~2~4~0~0)  \\[1pt]
             \normalsize  \bf S~u~m~m~e~r~~~2~0~0~6           \\[1pt]
      \end{centering}
    \vspace{6pt}
  }}
}

\vspace{20pt}

%===============================================================================

\normalsize

\begin{tabular}{lp{5.2in}}
%
\textbf{Instructor:}   
      & Matt Nickodemus
        \hspace{0.50ex} ---
        \hspace{0.50ex} MATH~362
        \hspace{1.75ex} nickodem@euclid.colorado.EDU
      \\[8pt]
%
\textbf{Lect. :} 
      & MTWTF 12:45-1:50      \\[-2pt]
%
\textbf{Office Hours:} 
      & MTWRF  11:30\/AM -- 12:30\/PM, or by appointment
      \\[10pt]
%
\textbf{Text:}
      & \textbf{\textsl{Multivariable Calculus }}
      \\[8pt]
%
\textbf{Calculators:}  
      & Can be helpful for homework and actually may be needed for some of the 
        problems but, to even the field for everybody, \textsl{they are not 
        allowed in tests!}
      \\[8pt]
%
\textbf{Prerequisite:}
      & Calculus 2.
      \\[8pt]
%
\textbf{Work Load:} 
      & This course will require \textsl{two hours work per day} outside class.
      \\[8pt]
%
\textbf{Grading:}      
      & $\triangleright$~Your course grade will be computed from:
      \\
      & $\circ\;$
        \makebox[6cm][s]{
        {\bf homework} 
        \dotfill
        {\bf 100 pts.}}
       \\
      & $\circ\;$
        \makebox[6cm][s]{
        {\bf 1 midterm exam}
        \dotfill 
        {\bf 100 pts.}}
      \\
      & $\circ\;$
        \makebox[6cm][s]{
        {\bf 1~final exam}
        \dotfill 
        {\bf 200 pts.}}
      \\[4pt]
      & $\triangleright$~\textbf{IF \& IW.}
%       It is not up to your instructor the granting of an IF or an IW grade.
        The Math Department has strict policies regarding these grades that
        require of \textsl{proper documentation} to accompany your 
        \textsl{petition} for an IF or an IW.
        These grades are \textbf{not} a substitute for timely filed 
        course-withdrawal form!
      \\[8pt]
%
\textbf{Homework:}    
      & Unless otherwise stated there will be a quiz over each homework assignment two days after it is given.       \\[8pt]
%
\textbf{Tests:}    
      & $\triangleright$~\textbf{Midterm Exam}.\/ 
        T.B.A.
            \\
      & $\triangleright$~\textbf{Final Exam}.\/ 
        T.B.A.      \\
       & $\triangleright$~Bring your CU student-ID to all your exams.
%     \\
%     & $\triangleright$~Try to write your tests \textbf{in ink!}
%       Tests written in pencil will \textbf{not} be accepted for revision due 
%       to alleged mistakes in the original grading.
      \\
      & $\triangleright$~The use of cellular phones, or any type of headset 
        during tests is strictly forbidden.
      \\
%     & $\triangleright$~Having \textsl{``the right concept''} 
%       is the first step in order to solve a test problem, but
%       \textsl{``the right concept''} alone is not enough to get credit for
%       it.
%       You will also need to make \textsl{considerable progress} towards the
%       \textsl{correct solution of the problem you were given}.
%     \\
      & $\triangleright$~Do not expect your test problems to be from the book!
      \\[8pt]
%
%\textbf{Important:}    
%      & Please be aware that you do not get a passing grade just for showing 
%        up for class or with written projects! 
%        \textsl{To pass this course you have to do well on your tests!}
%        Necessary conditions to achieve this goal are \textsl{attending class} 
%        and \textsl{doing your work regularly}.
%        Also, in tests you do not get points for \textsl{``how much you 
%        studied,''} 
%        or for \textsl{``how much you know,''} 
%        or \textsl{``for trying.''}
%        \textbf{\textsl{In tests you only get points for fully developed right 
%        solutions in writing to the test questions!}}
%        The \textsl{``right concept''} accounts only for a small part of the 
%        points a problem is worth.
%      \\[8pt]
%
\textbf{Web-pages:} 
      &  \texttt{http://registrar.colorado.edu/}
      \\ 
      &  \texttt{http://www.colorado.edu/sacs/disabilityservices/}
      \\ 
      &  \texttt{http://www.colorado.edu/policies/index.html}
      \\ 
      &  \texttt{http://www.colorado.edu/academics/honorcode/}
      \\[8pt]
\textbf{All the rest:} 
      & Please see me for any other problem not addressed in this set of 
        guidelines or for any questions you may have.
      \\[8pt]
%
\end{tabular}

%===============================================================================
\newpage
%===============================================================================

\textbf{\normalsize STUDENTS WITH DISABILITIES}

If you qualify for special accommodation because of disability, please submit
a letter from Disability Services to me early in the semester so that your
needs may be properly addressed.
Disability Services determines accommodations based on documented disabilities.
(\textbf{Disability Services Office, Willard~322, phone 303--492--8671},
see \texttt{http://www.colorado.edu/sacs/disabilityservices/}).

%-----------------------------------------------------------------------------

\medskip

\textbf{\normalsize RELIGIOUS OBLIGATIONS}

In case of conflict with a test, please let me know at least two weeks in
advance.
\\
See \texttt{http://www.colorado.edu/policies/index.html}
for the University's policy on religious obligations.

%-----------------------------------------------------------------------------

\medskip

\textbf{\normalsize STUDENT CLASSROOM AND COURSE-RELATED BEHAVIOR}

See \texttt{http://www.colorado.edu/policies/index.html} for the University's 
policy.

%-----------------------------------------------------------------------------

\medskip

\textbf{\normalsize UNIVERSITY'S HONOR CODE}

See \texttt{http://www.colorado.edu/academics/honorcode/}
for information on the University's honor code.

%=============================================================================

\bigskip \bigskip \bigskip \bigskip 

\centerline{\Large \bf HOMEWORK PROBLEMS UP TO JUNE 12}

\bigskip \bigskip 

\small
\begin{centering}
\begin{tabular}{||c|l|p{3.25in}|p{2.00in}||} \hline \hline
%------------------------------------------------------------------------------
              &          &                     &               \\
\bf hw \#        & \bf sctn & \centerline{\bf problems} & \centerline{\bf comments}
                              \\ \hline
%--------------------------------------------------------------------------
\multicolumn{4}{||c||}{\textbf{Capter 12}}
\\ \hline

%------------------------------------------------------------------------------
\bf  1 & 12.1  & 9, 11, 15,  21, 25, 33  &
               
\\ \hline
%------------------------------------------------------------------------------
\bf  1 & 12.2  & 5, 9, 11, 13, 17, 21, 23, 43, 45, 49   &
               
\\ \hline
%------------------------------------------------------------------------------
\bf  2 & 12.3  & 1, 3, 5, 13, 15, 25, 27, 28, 29, 32, 35  &
               
\\ \hline
%------------------------------------------------------------------------------
\bf  3 & 12.4  & 1, 3, 7, 11, 13, 15, 17, 21, 23, 25, 26, 29  &
               
\\ \hline
%------------------------------------------------------------------------------
\bf  4 & 12.5  & 3, 5, 7, 9, 11, 13, 17, 21, 23, 25, 29, 31, 33, 45, 49, 53  & 
                
\\ \hline
%------------------------------------------------------------------------------
\bf  5 & 12.6  & 1, 3, 7,  9, 11, 13, 15, 17, 19, 21, 25, 29, 33, 43, 47 &
                
\\ \hline
%------------------------------------------------------------------------------
\bf  6 & 12.7  & 1, 3, 7, 11, 13, 15, 19, 29, 37, 39, 43, 49  &
               
\\ \hline
\end{tabular}
\\
\end{centering}

%###############################################################################

\end{document}

%###############################################################################

